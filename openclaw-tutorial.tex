% OpenClaw チュートリアル
% LuaLaTeX で組版
% コンパイル: lualatex openclaw-tutorial.tex
\documentclass[a4paper,11pt]{ltjsarticle}

% ───────────────────────────────────────────
% パッケージ
% ───────────────────────────────────────────
\usepackage{luatexja-fontspec}
\usepackage{graphicx}
\usepackage{xcolor}
\usepackage{hyperref}
\usepackage{listings}
\usepackage{tcolorbox}
\usepackage{enumitem}
\usepackage{booktabs}
\usepackage{longtable}
\usepackage{geometry}
\usepackage{fancyhdr}
\usepackage{titlesec}

\tcbuselibrary{listings,breakable,skins}

% ───────────────────────────────────────────
% ページ設定
% ───────────────────────────────────────────
\geometry{margin=25mm}

% ───────────────────────────────────────────
% 色定義
% ───────────────────────────────────────────
\definecolor{clawred}{HTML}{E63946}
\definecolor{clawdark}{HTML}{1D3557}
\definecolor{clawblue}{HTML}{457B9D}
\definecolor{clawlight}{HTML}{F1FAEE}
\definecolor{codebg}{HTML}{F5F5F5}
\definecolor{codeframe}{HTML}{CCCCCC}

% ───────────────────────────────────────────
% ハイパーリンク設定
% ───────────────────────────────────────────
\hypersetup{
  colorlinks=true,
  linkcolor=clawdark,
  urlcolor=clawblue,
  citecolor=clawdark,
  bookmarksnumbered=true,
  pdftitle={OpenClaw チュートリアル},
  pdfauthor={河合勝彦},
}

% ───────────────────────────────────────────
% listings 設定
% ───────────────────────────────────────────
\lstset{
  basicstyle=\ttfamily\small,
  backgroundcolor=\color{codebg},
  frame=single,
  rulecolor=\color{codeframe},
  breaklines=true,
  breakatwhitespace=false,
  columns=fullflexible,
  keepspaces=true,
  showstringspaces=false,
  tabsize=2,
  xleftmargin=4mm,
  framexleftmargin=2mm,
}

\lstdefinelanguage{json}{
  morestring=[b]",
  literate=
    *{:}{{{\color{clawred}:}}}{1}
    {,}{{{\color{clawred},}}}{1}
    {\{}{{{\color{clawdark}\{}}}{1}
    {\}}{{{\color{clawdark}\}}}}{1}
    {[}{{{\color{clawdark}[}}}{1}
    {]}{{{\color{clawdark}]}}}{1},
}

% ───────────────────────────────────────────
% tcolorbox 定義
% ───────────────────────────────────────────
\newtcolorbox{notebox}{
  colback=clawlight,
  colframe=clawblue,
  fonttitle=\bfseries,
  title={NOTE},
  breakable,
  left=3mm, right=3mm, top=2mm, bottom=2mm,
}

\newtcolorbox{warnbox}{
  colback=yellow!10,
  colframe=clawred,
  fonttitle=\bfseries,
  title={WARNING},
  breakable,
  left=3mm, right=3mm, top=2mm, bottom=2mm,
}

\newtcolorbox{tipbox}{
  colback=green!5,
  colframe=green!50!black,
  fonttitle=\bfseries,
  title={TIP},
  breakable,
  left=3mm, right=3mm, top=2mm, bottom=2mm,
}

% ───────────────────────────────────────────
% セクション見出しスタイル
% ───────────────────────────────────────────
\titleformat{\section}
  {\Large\bfseries\color{clawdark}}{\thesection}{1em}{}
\titleformat{\subsection}
  {\large\bfseries\color{clawblue}}{\thesubsection}{1em}{}
\titleformat{\subsubsection}
  {\normalsize\bfseries\color{clawblue!80!black}}{\thesubsubsection}{1em}{}

% ───────────────────────────────────────────
% ヘッダ・フッタ
% ───────────────────────────────────────────
\pagestyle{fancy}
\fancyhf{}
\fancyhead[L]{\small\textcolor{clawdark}{OpenClaw チュートリアル}}
\fancyhead[R]{\small\textcolor{clawdark}{\leftmark}}
\fancyfoot[C]{\small\thepage}
\renewcommand{\headrulewidth}{0.4pt}

% ───────────────────────────────────────────
% タイトル情報
% ───────────────────────────────────────────
\title{%
  \vspace{-10mm}
  {\LARGE\bfseries\textcolor{clawdark}{OpenClaw チュートリアル}}\\[4mm]
  {\large オープンソース パーソナル AI アシスタント\\導入から実践まで}
}
\author{河合勝彦}
\date{2026年2月22日}

% ═══════════════════════════════════════════
\begin{document}
\maketitle
\thispagestyle{fancy}

% ───────────────────────────────────────────
\begin{abstract}
OpenClaw は、ローカル環境で動作するオープンソースのパーソナル AI アシスタントである。
WhatsApp・Telegram・Discord・Slack をはじめとする多数のチャットプラットフォームと統合し、
日常のタスク自動化を実現する。
本チュートリアルでは、インストールから基本設定、チャンネル連携、スキルの活用、
そして高度な自動化までを段階的に解説する。
\end{abstract}

\tableofcontents
\clearpage

% ═══════════════════════════════════════════
\section{OpenClaw とは}
% ═══════════════════════════════════════════

OpenClaw\footnote{\url{https://openclaw.ai/}}は、
MIT ライセンスで公開されているオープンソースのパーソナル AI アシスタントである。
以下の特徴を持つ。

\begin{itemize}
  \item \textbf{ローカル実行}:自分のマシン上で動作し、データのプライバシーを確保
  \item \textbf{マルチチャンネル対応}:WhatsApp、Telegram、Discord、Slack、Signal、iMessage、
        Google Chat、Microsoft Teams、Matrix、LINE など 15 以上のプラットフォームに対応
  \item \textbf{永続メモリ}:会話の文脈やユーザーの好みを学習し、24時間稼働
  \item \textbf{システムアクセス}:ブラウザ操作、ファイル読み書き、シェルコマンド実行が可能
  \item \textbf{拡張性}:ClawHub のスキルマーケットプレイスを通じてコミュニティ製スキルを導入可能
  \item \textbf{マルチモデル対応}:Anthropic Claude、OpenAI GPT、ローカルモデルを選択可能
\end{itemize}

\subsection{アーキテクチャ概要}

OpenClaw の中核は \textbf{Gateway} と呼ばれるプロセスである。
Gateway は WebSocket ベース(\texttt{ws://127.0.0.1:18789})で動作し、
セッション管理、チャンネル接続、ツール呼び出し、イベント処理を一元的に制御する。

\begin{center}
\begin{tcolorbox}[colback=white,colframe=clawdark,width=0.92\textwidth,title=アーキテクチャ]
\small
\begin{tabular}{ccc}
\begin{minipage}[t]{0.25\textwidth}
\centering
\textbf{チャットアプリ}\\[3mm]
\fbox{\small\begin{tabular}{l}
WhatsApp\\Telegram\\Discord\\Slack\\Signal\\...
\end{tabular}}\\[6mm]
\fbox{\small\begin{tabular}{l}
Dashboard\\{\ttfamily :18789}
\end{tabular}}
\end{minipage}
&
\begin{minipage}[t]{0.05\textwidth}
\centering
\vspace{10mm}
$\Longrightarrow$\\[20mm]
$\Longleftrightarrow$
\end{minipage}
&
\begin{minipage}[t]{0.28\textwidth}
\centering
\textbf{Gateway}\\[3mm]
\fbox{\small\begin{tabular}{l}
OpenClaw\\
Gateway {\ttfamily :18789}\\[2mm]
\textit{セッション管理}\\
\textit{ユーザー認証}\\
\textit{スキル読込}\\
\textit{メモリ管理}
\end{tabular}}
\end{minipage}
\begin{minipage}[t]{0.05\textwidth}
\centering
\vspace{10mm}
$\Longrightarrow$
\end{minipage}
\begin{minipage}[t]{0.25\textwidth}
\centering
\textbf{エージェント}\\[3mm]
\fbox{\small\begin{tabular}{l}
Pi Agent\\(RPC Mode)
\end{tabular}}\\[3mm]
$\Downarrow$\\[3mm]
\fbox{\small\begin{tabular}{l}
ツール群\\[1mm]
\textit{ブラウザ}\\
\textit{ファイル}\\
\textit{シェル}
\end{tabular}}
\end{minipage}
\end{tabular}
\end{tcolorbox}
\end{center}

% ═══════════════════════════════════════════
\section{動作環境の準備}
% ═══════════════════════════════════════════

\subsection{前提条件}

\begin{itemize}
  \item \textbf{OS}:macOS、Linux、または Windows(WSL2 経由)
  \item \textbf{Node.js}:バージョン 22 以上が必須
  \item \textbf{AI モデルの API キー}:Anthropic(Claude)または OpenAI(GPT)のいずれか
\end{itemize}

\begin{notebox}
Node.js のバージョンは \texttt{node --version} で確認できる。
v22 未満の場合は \url{https://nodejs.org/} から最新 LTS をインストールすること。
\end{notebox}

\subsection{Node.js のインストール(未導入の場合)}

\begin{lstlisting}[language=bash,title=nvm を使った Node.js のインストール]
# nvm のインストール
curl -o- https://raw.githubusercontent.com/nvm-sh/nvm/v0.40.1/install.sh | bash

# シェルの再読み込み
source ~/.bashrc   # bash の場合
# source ~/.zshrc  # zsh の場合

# Node.js 22 のインストールと有効化
nvm install 22
nvm use 22
node --version     # v22.x.x と表示されれば OK
\end{lstlisting}

% ═══════════════════════════════════════════
\section{インストール}
% ═══════════════════════════════════════════

インストール方法は 3 つ用意されている。

\subsection{方法 1:ワンライナー(推奨)}

\begin{lstlisting}[language=bash,title=macOS / Linux]
curl -fsSL https://openclaw.ai/install.sh | bash
\end{lstlisting}

\begin{lstlisting}[title=Windows (PowerShell)]
iwr -useb https://openclaw.ai/install.ps1 | iex
\end{lstlisting}

\subsection{方法 2:npm によるインストール}

\begin{lstlisting}[language=bash]
npm install -g openclaw@latest
\end{lstlisting}

\subsection{方法 3:ソースからビルド(開発者向け)}

\begin{lstlisting}[language=bash]
git clone https://github.com/openclaw/openclaw.git
cd openclaw
pnpm install
pnpm build
pnpm openclaw onboard --install-daemon
\end{lstlisting}

\begin{tipbox}
ソースビルドでは \texttt{pnpm} が必要である。
\texttt{npm install -g pnpm} で導入できる。
\end{tipbox}

% ═══════════════════════════════════════════
\section{初期セットアップ}
% ═══════════════════════════════════════════

\subsection{オンボーディングウィザードの実行}

インストール後、以下のコマンドで対話型セットアップウィザードを起動する。

\begin{lstlisting}[language=bash]
openclaw onboard --install-daemon
\end{lstlisting}

ウィザードでは以下の項目を順に設定する。

\begin{enumerate}
  \item \textbf{Gateway の設定}:ポート番号(デフォルト 18789)とセキュリティ
  \item \textbf{ワークスペースの構成}:作業ディレクトリの場所
  \item \textbf{AI モデルの選択}:Claude / GPT / ローカルモデル
  \item \textbf{認証設定}:API キーまたは OAuth の設定
  \item \textbf{チャンネルの接続}:最初のメッセージングプラットフォームの追加
  \item \textbf{スキルのインストール}:初期スキルの選択
\end{enumerate}

\begin{notebox}
オンボーディングは通常 10 分以内に完了する。
各ステップは後からでも \texttt{openclaw configure} で変更可能である。
\end{notebox}

\subsection{Gateway の起動確認}

\begin{lstlisting}[language=bash]
# Gateway のステータス確認
openclaw gateway status

# フォアグラウンドで起動(デバッグ時に便利)
openclaw gateway --port 18789 --verbose
\end{lstlisting}

\subsection{Dashboard(管理画面)へのアクセス}

\begin{lstlisting}[language=bash]
openclaw dashboard
\end{lstlisting}

ブラウザで \url{http://127.0.0.1:18789/} にアクセスすると、
管理画面(Dashboard)が表示される。
ここから以下の操作が可能である。

\begin{itemize}
  \item セッションの監視と管理
  \item チャンネルの追加・設定
  \item スキルの有効化・無効化
  \item メモリの確認
  \item ログの閲覧
\end{itemize}

% ═══════════════════════════════════════════
\section{設定ファイル}
% ═══════════════════════════════════════════

OpenClaw の主要な設定は \texttt{\~{}/.openclaw/openclaw.json} に記述する。

\subsection{基本設定の例}

\begin{lstlisting}[language=json,title=\~{}/.openclaw/openclaw.json]
{
  "model": {
    "provider": "anthropic",
    "name": "claude-sonnet-4-6"
  },
  "agents": {
    "defaults": {
      "workspace": "~/.openclaw/workspace"
    }
  },
  "gateway": {
    "port": 18789
  },
  "skills": {
    "entries": {}
  }
}
\end{lstlisting}

\subsection{主要な設定項目}

\begin{longtable}{lp{85mm}}
\toprule
\textbf{キー} & \textbf{説明} \\
\midrule
\endhead
\texttt{model.provider} & モデルプロバイダ(\texttt{anthropic}, \texttt{openai} など) \\
\texttt{model.name} & 使用するモデル名 \\
\texttt{gateway.port} & Gateway のポート番号 \\
\texttt{agents.defaults.workspace} & ワークスペースのパス \\
\texttt{skills.entries} & スキルごとの設定(API キー等) \\
\texttt{skills.load.extraDirs} & 追加のスキル読込ディレクトリ \\
\texttt{skills.load.watch} & スキルファイルの変更監視(デフォルト \texttt{true}) \\
\bottomrule
\end{longtable}

\subsection{環境変数}

設定ファイルの代わりに環境変数でも制御できる。

\begin{longtable}{lp{85mm}}
\toprule
\textbf{変数名} & \textbf{説明} \\
\midrule
\endhead
\texttt{OPENCLAW\_HOME} & ホームディレクトリの上書き \\
\texttt{OPENCLAW\_STATE\_DIR} & ステートディレクトリの上書き \\
\texttt{OPENCLAW\_CONFIG\_PATH} & 設定ファイルパスの上書き \\
\bottomrule
\end{longtable}

% ═══════════════════════════════════════════
\section{チャンネルの設定}
% ═══════════════════════════════════════════

OpenClaw の最大の特長は、日常的に使うチャットアプリを通じて AI アシスタントと
対話できる点にある。本章では主要なチャンネルの設定方法を解説する。

\subsection{対応チャンネル一覧}

\begin{longtable}{lll}
\toprule
\textbf{チャンネル} & \textbf{接続方式} & \textbf{備考} \\
\midrule
\endhead
WhatsApp & WhatsApp Web & QR コード認証 \\
Telegram & Bot API & BotFather でトークン取得 \\
Discord & Bot API & Developer Portal で設定 \\
Slack & Bot API & Slack App を作成 \\
Signal & Signal CLI & 電話番号認証 \\
iMessage & BlueBubbles & macOS のみ \\
Google Chat & Workspace API & Google Cloud 設定 \\
Microsoft Teams & Bot Framework & Azure 設定 \\
Matrix & Bot SDK & ホームサーバー接続 \\
LINE & Messaging API & LINE Developers で設定 \\
IRC & IRC プロトコル & サーバー接続 \\
\bottomrule
\end{longtable}

\subsection{Telegram の設定例}

Telegram は最も簡単に設定できるチャンネルの一つである。

\begin{enumerate}
  \item Telegram で \texttt{@BotFather} にメッセージを送信
  \item \texttt{/newbot} コマンドでボットを作成
  \item ボット名とユーザー名を設定
  \item 発行された API トークンをコピー
  \item OpenClaw に設定を追加:
\end{enumerate}

\begin{lstlisting}[language=bash,title=CLI でチャンネルを追加]
openclaw channels add telegram
# 対話型プロンプトに従い、API トークンを入力
\end{lstlisting}

または、設定ファイルに直接記述する。

\begin{lstlisting}[language=json,title=openclaw.json にチャンネルを追加]
{
  "channels": {
    "telegram": {
      "enabled": true,
      "token": "YOUR_BOT_TOKEN"
    }
  }
}
\end{lstlisting}

\begin{warnbox}
API トークンは秘密情報である。Git リポジトリにコミットしないよう注意すること。
環境変数での管理を推奨する。
\end{warnbox}

\subsection{WhatsApp の設定例}

\begin{enumerate}
  \item \texttt{openclaw channels add whatsapp} を実行
  \item ターミナルに表示される QR コードを WhatsApp アプリでスキャン
  \item 接続が確立されると、OpenClaw がメッセージを受信開始
\end{enumerate}

\begin{notebox}
WhatsApp は WhatsApp Web プロトコルを使用するため、
スマートフォン側の WhatsApp が起動している必要がある。
\end{notebox}

\subsection{DM セキュリティ(ペアリングポリシー)}

デフォルトでは、未知の送信者からの DM にはペアリングコードが要求される。
これにより、不正なメッセージの処理を防止する。

\begin{lstlisting}[language=bash,title=ペアリングの管理]
# ペアリングコードの生成
openclaw pairing create

# 特定のユーザーを許可リストに追加
openclaw pairing allow +819012345678
\end{lstlisting}

% ═══════════════════════════════════════════
\section{基本的な使い方}
% ═══════════════════════════════════════════

\subsection{チャットでの対話}

設定済みのチャンネル(Telegram、WhatsApp 等)から
ボットにメッセージを送信するだけで AI アシスタントと対話できる。

\subsubsection{基本コマンド}

チャット内で使えるスラッシュコマンドを以下に示す。

\begin{longtable}{lp{80mm}}
\toprule
\textbf{コマンド} & \textbf{説明} \\
\midrule
\endhead
\texttt{/status} & セッションの状態表示 \\
\texttt{/new} または \texttt{/reset} & セッションのリセット(コンテキストクリア) \\
\texttt{/think <level>} & 推論の深さを設定(\texttt{low}, \texttt{medium}, \texttt{high}) \\
\texttt{/compact} & 会話を要約して圧縮 \\
\texttt{/restart} & Gateway の再起動 \\
\bottomrule
\end{longtable}

\subsection{CLI からの操作}

チャットアプリを使わず、CLI から直接操作することも可能である。

\begin{lstlisting}[language=bash]
# メッセージの送信
openclaw message send --to +819012345678 --message "こんにちは"

# エージェントに直接質問
openclaw agent --message "明日の天気を教えて" --thinking high

# TUI(ターミナル UI)の起動
openclaw tui
\end{lstlisting}

\subsection{Dashboard からの操作}

\texttt{openclaw dashboard} で起動するウェブ UI からも操作できる。
Dashboard では以下が可能である。

\begin{itemize}
  \item リアルタイムのセッション監視
  \item WebChat を使った直接対話
  \item スキルの管理
  \item ログの閲覧とフィルタリング
  \item チャンネルの状態確認
\end{itemize}

% ═══════════════════════════════════════════
\section{スキルの管理}
% ═══════════════════════════════════════════

スキルは OpenClaw の機能を拡張するモジュールである。
スキルは「ツールの組み合わせ方をエージェントに教えるテキストブック」であり、
新しい権限を付与するものではない。

\subsection{スキルの種類と読込順序}

スキルは以下の 3 箇所から読み込まれる(優先度順)。

\begin{enumerate}
  \item \textbf{ワークスペーススキル}:\texttt{<workspace>/skills/}(エージェント固有)
  \item \textbf{マネージドスキル}:\texttt{\~{}/.openclaw/skills/}(全エージェント共有)
  \item \textbf{バンドルスキル}:OpenClaw 本体に同梱
\end{enumerate}

\subsection{ClawHub からのインストール}

ClawHub\footnote{\url{https://clawhub.com}} はコミュニティ製スキルの公開レジストリである。

\begin{lstlisting}[language=bash]
# スキルのインストール
clawhub install <skill-slug>

# インストール済みスキルの一覧
clawhub list

# 全スキルの更新
clawhub update --all

# スキルのアンインストール
clawhub uninstall <skill-slug>
\end{lstlisting}

\begin{lstlisting}[language=bash,title=CLI でスキルを管理]
# 利用可能なスキルの確認
openclaw skills list --eligible

# スキルの詳細情報
openclaw skills info <skill-name>

# スキルの動作確認
openclaw skills check
\end{lstlisting}

\subsection{スキルの設定}

スキルごとの設定は \texttt{openclaw.json} の \texttt{skills.entries} に記述する。

\begin{lstlisting}[language=json,title=スキル設定の例]
{
  "skills": {
    "entries": {
      "spotify": {
        "enabled": true,
        "apiKey": "YOUR_SPOTIFY_API_KEY",
        "env": {
          "SPOTIFY_CLIENT_ID": "abc123",
          "SPOTIFY_CLIENT_SECRET": "secret456"
        }
      },
      "gmail": {
        "enabled": true
      },
      "github": {
        "enabled": true,
        "apiKey": "ghp_xxxxxxxxxxxx"
      }
    }
  }
}
\end{lstlisting}

\subsection{カスタムスキルの作成}

独自のスキルを作成するには、ワークスペースの \texttt{skills/} ディレクトリに
\texttt{SKILL.md} ファイルを配置する。

\begin{lstlisting}[title=\~{}/.openclaw/workspace/skills/my-skill/SKILL.md]
---
name: my-custom-skill
description: 独自のカスタムスキル
user-invocable: true
metadata: {"openclaw":{"emoji":"🔧","requires":{"bins":["curl"]}}}
---

# My Custom Skill

このスキルは以下のことができます:

1. 特定の API からデータを取得する
2. 取得したデータを整形してユーザーに返す

## 使い方

ユーザーが「データを取得して」と言ったら、
以下の手順で処理してください:

1. curl を使って API にリクエストを送信
2. レスポンスを JSON としてパース
3. 必要な情報を抽出してユーザーに提示
\end{lstlisting}

\begin{notebox}
スキルのファイル変更は自動検知される(\texttt{skills.load.watch: true})。
新しいセッションで変更が反映される。
\end{notebox}

\begin{warnbox}
ClawHub のサードパーティスキルは、インストール前に内容を確認すること。
2026年2月時点で悪意のあるスキルの報告例がある。
\end{warnbox}

% ═══════════════════════════════════════════
\section{ワークスペースとメモリ}
% ═══════════════════════════════════════════

\subsection{ワークスペース構造}

デフォルトのワークスペースは \texttt{\~{}/.openclaw/workspace/} に配置され、
以下のファイルでエージェントの挙動をカスタマイズできる。

\begin{longtable}{lp{80mm}}
\toprule
\textbf{ファイル} & \textbf{役割} \\
\midrule
\endhead
\texttt{AGENTS.md} & エージェントの定義と設定 \\
\texttt{SOUL.md} & エージェントの性格・話し方の設定 \\
\texttt{TOOLS.md} & 利用可能なツールのドキュメント \\
\texttt{skills/} & カスタムスキルの格納ディレクトリ \\
\bottomrule
\end{longtable}

\subsection{SOUL.md によるパーソナリティ設定}

\texttt{SOUL.md} を編集することで、アシスタントの応答スタイルを自由にカスタマイズできる。

\begin{lstlisting}[title=SOUL.md の例]
あなたは親切で丁寧な日本語アシスタントです。

- 敬語を使って応答してください
- 技術的な質問には具体的なコード例を添えてください
- 分からないことは正直に「分かりません」と答えてください
- ユーモアを交えつつ、正確な情報を提供してください
\end{lstlisting}

\subsection{メモリシステム}

OpenClaw は永続メモリを備えており、
ユーザーの好み、過去の会話、よく使うパターンなどを自動的に記憶する。

\begin{lstlisting}[language=bash]
# メモリの確認
openclaw memory

# メモリの管理(Dashboard からも可能)
openclaw dashboard
\end{lstlisting}

% ═══════════════════════════════════════════
\section{自動化機能}
% ═══════════════════════════════════════════

OpenClaw は高度な自動化機能を備えている。

\subsection{Cron ジョブ}

定期的なタスクを cron 式で設定できる。

\begin{lstlisting}[language=bash]
# cron ジョブの管理
openclaw cron list
openclaw cron add "0 9 * * *" "今日の予定を教えて"
openclaw cron remove <job-id>
\end{lstlisting}

チャット内から設定することも可能である。

\begin{quote}
\textit{「毎朝 9 時にニュースの要約を送って」}
\end{quote}

\subsection{Webhook}

外部サービスからのイベントを受信して処理できる。

\begin{lstlisting}[language=bash]
# Webhook の管理
openclaw hooks list
openclaw hooks add --url /my-webhook --action "通知を処理"
\end{lstlisting}

\subsection{Gmail 連携}

Gmail Pub/Sub を設定することで、メールの受信をトリガーにした自動処理が可能になる。

% ═══════════════════════════════════════════
\section{ブラウザ操作}
% ═══════════════════════════════════════════

OpenClaw は Chrome/Chromium を制御し、ウェブブラウジングやフォーム入力を自動化できる。

\begin{lstlisting}[language=bash]
# ブラウザの起動
openclaw browser
\end{lstlisting}

チャットで以下のような依頼が可能である。

\begin{itemize}
  \item 「\texttt{example.com} を開いてスクリーンショットを撮って」
  \item 「GitHub の Issue \#42 の内容を確認して」
  \item 「このフォームに以下の内容を入力して送信して」
\end{itemize}

\begin{warnbox}
ブラウザ操作は強力な機能であるため、
サンドボックスモードでの実行を推奨する。
\end{warnbox}

% ═══════════════════════════════════════════
\section{リモートアクセス}
% ═══════════════════════════════════════════

自宅や VPS 上の Gateway に外出先からアクセスする方法を説明する。

\subsection{Tailscale によるアクセス}

Tailscale\footnote{\url{https://tailscale.com/}} を使えば、
安全なリモートアクセスが簡単に実現できる。

\begin{lstlisting}[language=bash]
# Tailscale Serve(tailnet 内のみ)
tailscale serve --bg 18789

# Tailscale Funnel(公開アクセス、パスワード認証必須)
tailscale funnel --bg 18789
\end{lstlisting}

\subsection{SSH トンネル}

SSH トンネルを使ったリモートアクセスも可能である。

\begin{lstlisting}[language=bash]
# リモートサーバーの Gateway にトンネル接続
ssh -L 18789:localhost:18789 user@remote-server
\end{lstlisting}

% ═══════════════════════════════════════════
\section{セキュリティ}
% ═══════════════════════════════════════════

\subsection{基本方針}

\begin{itemize}
  \item インバウンドの DM は常に「信頼できない入力」として扱う
  \item デフォルトのペアリングポリシー(\texttt{dmPolicy="pairing"})を維持する
  \item サードパーティスキルのインストール前に内容を確認する
  \item API キーやトークンは環境変数で管理し、設定ファイルに直接記述しない
\end{itemize}

\subsection{サンドボックスモード}

信頼度の低いツールやスキルを実行する際は、
サンドボックス(Docker コンテナ)内で実行できる。

\begin{lstlisting}[language=json,title=サンドボックスの設定例]
{
  "agents": {
    "defaults": {
      "sandbox": {
        "docker": {
          "enabled": true,
          "setupCommand": "apt-get update && apt-get install -y curl"
        }
      }
    }
  }
}
\end{lstlisting}

\subsection{昇格モード}

特定のセッションで管理者権限が必要な場合、
昇格モードを有効にできる(許可リストに登録済みのユーザーのみ)。

\begin{lstlisting}[title=チャット内での昇格モード切替]
/elevated on    # 昇格モード有効化
/elevated off   # 昇格モード無効化
\end{lstlisting}

% ═══════════════════════════════════════════
\section{トラブルシューティング}
% ═══════════════════════════════════════════

\subsection{診断コマンド}

\begin{lstlisting}[language=bash]
# システム診断
openclaw doctor

# ヘルスチェック
openclaw health

# ログの確認
openclaw logs

# 詳細ログ付きで Gateway を起動
openclaw gateway --port 18789 --verbose
\end{lstlisting}

\subsection{よくある問題と対処}

\begin{longtable}{p{55mm}p{75mm}}
\toprule
\textbf{症状} & \textbf{対処法} \\
\midrule
\endhead
Gateway が起動しない &
\texttt{openclaw doctor} で診断。ポート競合を確認 \\
\addlinespace
チャンネルが接続できない &
API トークンの有効性を確認。\texttt{openclaw channels} で状態を確認 \\
\addlinespace
スキルが読み込まれない &
\texttt{openclaw skills list --eligible} で確認。
\texttt{SKILL.md} の \texttt{requires} 条件を確認 \\
\addlinespace
応答が遅い &
\texttt{/think low} で推論レベルを下げる。
モデルのフェイルオーバー設定を確認 \\
\addlinespace
メモリが反映されない &
\texttt{/new} でセッションをリセット \\
\bottomrule
\end{longtable}

% ═══════════════════════════════════════════
\section{CLI コマンドリファレンス}
% ═══════════════════════════════════════════

主要な CLI コマンドの一覧を示す。

\begin{longtable}{lp{80mm}}
\toprule
\textbf{コマンド} & \textbf{説明} \\
\midrule
\endhead
\texttt{openclaw onboard} & オンボーディングウィザードの起動 \\
\texttt{openclaw gateway} & Gateway の起動・管理 \\
\texttt{openclaw gateway status} & Gateway の状態確認 \\
\texttt{openclaw dashboard} & Dashboard(Web UI)の起動 \\
\texttt{openclaw tui} & ターミナル UI の起動 \\
\texttt{openclaw agent --message} & エージェントへの直接メッセージ \\
\texttt{openclaw message send} & メッセージの送信 \\
\texttt{openclaw channels} & チャンネルの管理 \\
\texttt{openclaw skills} & スキルの管理 \\
\texttt{openclaw configure} & 設定の変更 \\
\texttt{openclaw doctor} & システム診断 \\
\texttt{openclaw health} & ヘルスチェック \\
\texttt{openclaw logs} & ログの表示 \\
\texttt{openclaw memory} & メモリの管理 \\
\texttt{openclaw cron} & Cron ジョブの管理 \\
\texttt{openclaw hooks} & Webhook の管理 \\
\texttt{openclaw browser} & ブラウザの管理 \\
\texttt{openclaw update} & OpenClaw の更新 \\
\texttt{openclaw update --channel beta} & ベータ版への切替 \\
\texttt{openclaw uninstall} & アンインストール \\
\texttt{openclaw reset} & システムのリセット \\
\bottomrule
\end{longtable}

% ═══════════════════════════════════════════
\section{実践例:日常タスクの自動化}
% ═══════════════════════════════════════════

ここでは OpenClaw を使った具体的なユースケースを紹介する。

\subsection{例 1:朝のブリーフィング自動化}

毎朝 7 時に、天気予報・本日の予定・未読メールの要約を
Telegram に送信する設定を行う。

\begin{enumerate}
  \item Telegram チャンネルを設定(第\ref{sec:channel}節を参照……前述の通り)
  \item 必要なスキル(天気、カレンダー、Gmail)をインストール
  \item Cron ジョブを設定
\end{enumerate}

\begin{lstlisting}[language=bash]
clawhub install weather
clawhub install google-calendar
clawhub install gmail
openclaw cron add "0 7 * * *" \
  "おはようございます。今日の天気、予定、未読メールを要約してください"
\end{lstlisting}

\subsection{例 2:GitHub Issue の自動トリアージ}

Webhook を使って新規 Issue を自動的に分類する。

\begin{lstlisting}[language=bash]
clawhub install github
openclaw hooks add --url /github-webhook \
  --action "新しい Issue をラベル付けして優先度を判定"
\end{lstlisting}

\subsection{例 3:ファイル管理}

チャットから直接ファイルの操作を指示する。

\begin{quote}
\textit{「Downloads フォルダの PDF ファイルを日付別に整理して」}\\
\textit{「このスプレッドシートのデータを集計して結果を教えて」}\\
\textit{「プロジェクトの README.md を更新して」}
\end{quote}

% ═══════════════════════════════════════════
\section{モバイルアプリ・デスクトップアプリ}
% ═══════════════════════════════════════════

OpenClaw は以下のコンパニオンアプリを提供している。

\begin{longtable}{lp{80mm}}
\toprule
\textbf{プラットフォーム} & \textbf{特徴} \\
\midrule
\endhead
macOS & メニューバーアプリ、Canvas(ビジュアルワークスペース)、
Voice Wake(音声起動) \\
iOS & カメラ、位置情報などデバイス機能へのアクセス \\
Android & カメラ、位置情報、通知など \\
Linux & デスクトップアプリ \\
\bottomrule
\end{longtable}

\subsection{音声機能}

Voice Wake と Talk Mode を有効にすると、
音声による常時待ち受けと対話が可能になる(ElevenLabs 連携)。

% ═══════════════════════════════════════════
\section{まとめ}
% ═══════════════════════════════════════════

本チュートリアルでは、OpenClaw の導入から実践的な活用までを解説した。
主要なポイントを以下にまとめる。

\begin{enumerate}
  \item \textbf{インストール}は \texttt{curl} または \texttt{npm} で簡単に行える
  \item \texttt{openclaw onboard} で対話型セットアップが完了する
  \item 好みの\textbf{チャットアプリ}をチャンネルとして接続し、自然言語で対話できる
  \item \textbf{ClawHub} のスキルで機能を拡張できる
  \item \textbf{Cron}、\textbf{Webhook}、\textbf{ブラウザ操作}で高度な自動化が実現できる
  \item \textbf{セキュリティ}ではペアリングポリシーとサンドボックスの活用が重要である
\end{enumerate}

より詳しい情報は以下を参照されたい。

\begin{itemize}
  \item 公式ドキュメント:\url{https://docs.openclaw.ai/}
  \item GitHub リポジトリ:\url{https://github.com/openclaw/openclaw}
  \item ClawHub(スキルレジストリ):\url{https://clawhub.com}
  \item コミュニティスキル集:\url{https://github.com/VoltAgent/awesome-openclaw-skills}
\end{itemize}

\end{document}
